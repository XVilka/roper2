% Created 2018-07-20 Fri 20:37
% Intended LaTeX compiler: pdflatex
\documentclass[11pt]{article}
\usepackage[utf8]{inputenc}
\usepackage[T1]{fontenc}
\usepackage{graphicx}
\usepackage{grffile}
\usepackage{longtable}
\usepackage{wrapfig}
\usepackage{rotating}
\usepackage[normalem]{ulem}
\usepackage{amsmath}
\usepackage{textcomp}
\usepackage{amssymb}
\usepackage{capt-of}
\usepackage{hyperref}
\usepackage{listings}
\usepackage[hidelinks]{hyperref}
%\usepackage[a-1b]{pdfx}
\usepackage[acronym]{glossaries}
\usepackage[scaled=0.85]{beramono}
\usepackage{authorindex}
\let\cite=\aicite
%\usepackage{algorithmic}
%\usepackage[chapter]{algorithm}
%\renewcommand{\listalgorithmname}{ }
\usepackage{color}
\usepackage{needspace}
\usepackage{multicol,calc,listings}
\lstset{
  basicstyle=\ttfamily,
  keepspaces=true,
  columns=fullflexible,
  escapeinside={/*|}{*|/}
}
\lstset{basicstyle=\footnotesize\ttfamily}
\usepackage{booktabs,url}
%  For formal tables
%{\ttfamily \hyphenchar\the\font=`\-} 
\usepackage{makeidx}
\makeindex

%\setcounter{secnumdepth}{5}
% Render all text in math sections monospaced
%\everymath{\mathtt{\xdef\tmp{\fam\the\fam\relax}\aftergroup\tmp}}
%\everydisplay{\mathtt{\xdef\tmp{\fam\the\fam\relax}\aftergroup\tmp}}
%\setbox0\hbox{$ $} 

%\usepackage{quotchap}
\makenoidxglossaries
\loadglsentries{../lit-glossary}

%\usepackage[all]{nowidow}
%\usepackage{imakeidx}
%\makeindex[title={}]

%\widowpenalty10000
%\clubpenalty10000

%\usepackage[lmargin=3.81cm,tmargin=2.54cm,bmargin=2.54cm,rmargin=2.54cm]{geometry}


\author{oblivia simplex}
\date{\today}
\title{Selection}
\hypersetup{
 pdfauthor={oblivia simplex},
 pdftitle={Selection},
 pdfkeywords={},
 pdfsubject={},
 pdfcreator={Emacs 25.2.2 (Org mode 9.1.13)}, 
 pdflang={English}}
\begin{document}

\maketitle
\tableofcontents


\section{Selection on a Stream}
\label{sec:org483caca}

One of the interesting design decisions that crystallized in engineering
this iteration of ROPER has been to treat the population as a cyclical
"stream", rather than as a mutable collection. 

The stream originates in with the seeder, proceeds through the hatchery,
on to the evaluator, and then to the selection and breeding actors, without
any need to synchronize a mutable population vector. What makes this feasible
is the way that Rust handles the \texttt{Send} trait: all that's transferred when a
\texttt{Creature} is sent across a channel is the deed for ownership. This operation
is no slower than indexing into a vector, practically speaking. 

The only real speedbump lies with the selection actor. Sticking with tournament
selection for the time being, we want to retain some capacity to select the
combatants in a tournament \emph{at random}. But randomly selecting from a stream
seems to require first collecting the incoming elements into a buffer. 

So, let there be a buffer. The selector will wait until \texttt{n} creatures have
arrived through the channel, and then perform tournament selection on that
buffer. Some number \texttt{tsize} of those creatures will be chosen for a tournament
-- perhaps several tournaments, in parallel. It will \emph{take} \texttt{tsize} creatures, 
on a secondary channel, then return \texttt{tsize} back, but of those \texttt{tsize}, \texttt{tsize/2}
will be the winners of the tournament, and \texttt{tsize/2} will be newborns.

\subsection{Spawning the selector/breeder}
\label{sec:orgc859780}

\lstset{language=rust,label=org7a49500,caption= ,captionpos=b,numbers=none}
\begin{lstlisting}
  <<bring dependencies into scope>>
    pub fn spawn_breeder(
        window_size: usize,
        rng_seed: RngSeed,
    ) -> (Sender<Creature>, Receiver<Creature>, JoinHandle<()>) {
        let (from_breeder_tx, from_breeder_rx) = channel();
        let (into_breeder_tx, into_breeder_rx) = channel();

        let window = Arc::new(RefCell::new(Vec::with_capacity(window_size+1)));
        let mut rng_seed = rng_seed.clone();

        let sel_handle = spawn(move || {
            let window = window.clone();
            for creature in into_breeder_rx {
                let mut window = window.borrow_mut();
                window.push(creature);
                if window.len() >= window_size {
                    /* then it's time to select breeders */
                    rng_seed = tournament(&mut window, rng_seed);
                    /* now send them back. new children will have replaced the dead */
                    for creature in window {
                        from_breeder_rx.send(creature)
                    }
                    /* Now, flush the window out so that it can refill. */
                    window.truncate(0);
                }
            }
        });

        (into_breeder_tx, from_breeder_rx, sel_handle)
    }
<<perform selection and mating>>
\end{lstlisting}

\subsection{Selection functions}
\label{sec:orge1d6443}

To work with the form of homologous crossover implemented in the
\texttt{emu::crossover} module, we may wish to use simple mate selection
algorithm, which increases the likelihood that mating pairs will
have "compatible" crossover masks. But this is a probabilistically
delicate operation. We don't want to create a perverse incentive
that will incline the population towards crossover masks that consist
entirely of \texttt{1} bits (and so which are \emph{maximally compatible} with other
masks), simply for the sake of increasing their likelihood of being
chosen for tournaments. 

On the other hand, this incentive will only turn out to be "perverse"
if it overwhelms the selective pressure (which we have theoretically
grounded reasons to expect) for sparse crossover masks. It could turn
out to be a useful, countervailing pressure that inclines the masks
to be as dense as possible, without losing the benefits of sparseness.
(The benefit of a sparse crossover mask, of course, is that it reduces
the probability of destructive crossover.)

\lstset{language=rust,label=org629c022,caption= ,captionpos=b,numbers=none}
\begin{lstlisting}
fn xover_compat(c1: &Creature, c2: &Creature) -> usize {
    (c1.genome.xbits & c2.genome.xbits).count_ones()
}
\end{lstlisting}

The static variable \texttt{MATE\_SELECTION\_FACTOR} will be used\ldots{}

\lstset{language=rust,label=org4946d0e,caption= ,captionpos=b,numbers=none}
\begin{lstlisting}
fn tournament(selection_window: &mut Vec<Creature>,
              seed: RngSeed) -> RngSeed {
    let mut rng = Isaac64Rng::from_seed(RngSeed);
    /* note: seed creation should probably be its own utility function */
    let mut new_seed: [u8; 32] = [0; 32];
    for i in 0..32 { new_seed[i] = rng.gen::<u8>() }

    assert!(*TSIZE * *MATE_SELECTION_FACTOR <= selection_window.len());
    let indices = rand::seq::sample_indices(&mut rng,
                                            selection_window.len(),
                                            *TSIZE * *MATE_SELECTION_FACTOR);
    /* TODO: take n times as many combatants as needed, then winnow
     * out those least compatible with first combatant's crossover mask
     */
    let compatkey = |i| {
        64 - xover_compat(&selection_window[0], &selection_window[i])
    }
    indices.sort_by_key(compatkey);
    /* now drop the least compatible from consideration */
    indices.truncate(*TSIZE);

    let fitkey = |i| {
        &selection_window[i].fitness
    }
    /* now, sort the remaining indices by the fitness of their creatures */
    /* TODO -- we need a pareto sorting function */
    indices.sort_by_key(fitkey);
    /* and choose the parents and the fallen */
    // *TSIZE must be >= 4.
    let (p0, p1) = (indices[*TSIZE-1], indices[*TSIZE-2]);
    let (d0, d1) = (indices[0], indices[1]);
    let (mother, father) = (&selection_window[p0], &selection_window[p1]);
    let offspring = homologous_crossover(&mother, &father, &mut rng);
    /* now, place the offspring back in the population by inserting them
     * into the selection window
     */
    selection_window[d0] = offspring[0];
    selection_window[d1] = offspring[1];

    new_seed
}
\end{lstlisting}
\end{document}