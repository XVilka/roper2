% Created 2018-10-16 Tue 08:57
% Intended LaTeX compiler: pdflatex
\documentclass[11pt]{article}
\usepackage[utf8]{inputenc}
\usepackage[T1]{fontenc}
\usepackage{graphicx}
\usepackage{grffile}
\usepackage{longtable}
\usepackage{wrapfig}
\usepackage{rotating}
\usepackage[normalem]{ulem}
\usepackage{amsmath}
\usepackage{textcomp}
\usepackage{amssymb}
\usepackage{capt-of}
\usepackage{hyperref}
\usepackage{listings}
\usepackage[hidelinks]{hyperref}
%\usepackage[a-1b]{pdfx}
\usepackage[acronym]{glossaries}
\usepackage[scaled=0.85]{beramono}
\usepackage{authorindex}
\let\cite=\aicite
%\usepackage{algorithmic}
%\usepackage[chapter]{algorithm}
%\renewcommand{\listalgorithmname}{ }
\usepackage{color}
\usepackage{needspace}
\usepackage{multicol,calc,listings}
\lstset{
  basicstyle=\ttfamily,
  keepspaces=true,
  columns=fullflexible,
  escapeinside={/*|}{*|/}
}
\lstset{basicstyle=\footnotesize\ttfamily}
\usepackage{booktabs,url}
%  For formal tables
%{\ttfamily \hyphenchar\the\font=`\-} 
\usepackage{makeidx}
\makeindex

%\setcounter{secnumdepth}{5}
% Render all text in math sections monospaced
%\everymath{\mathtt{\xdef\tmp{\fam\the\fam\relax}\aftergroup\tmp}}
%\everydisplay{\mathtt{\xdef\tmp{\fam\the\fam\relax}\aftergroup\tmp}}
%\setbox0\hbox{$ $} 

%\usepackage{quotchap}
\makenoidxglossaries
\loadglsentries{../lit-glossary}

%\usepackage[all]{nowidow}
%\usepackage{imakeidx}
%\makeindex[title={}]

%\widowpenalty10000
%\clubpenalty10000

%\usepackage[lmargin=3.81cm,tmargin=2.54cm,bmargin=2.54cm,rmargin=2.54cm]{geometry}


\date{\today}
\title{The ROPER 2 Hatchery}
\hypersetup{
 pdfauthor={},
 pdftitle={The ROPER 2 Hatchery},
 pdfkeywords={},
 pdfsubject={},
 pdfcreator={Emacs 26.1 (Org mode 9.1.14)}, 
 pdflang={English}}
\begin{document}

\maketitle
\tableofcontents


\section{The Hatchery}
\label{sec:org6961ccb}

The hatchery module of \gls{roper2} consists of two logical components: a
mechanism for performing "\gls{rop} chain embryogenesis", which maps genotypes to
phenotypes so as to prepare them for fitness evaluation and selection (\S
\ref{org2e86c3c}), and a mechanism to handle the concurrency plumbing for the
system -- setting up multiple \texttt{unicorn} emulator instances on separate, looping
threads, with which the rest the system can communicate through a network of
\texttt{channels} (\S \ref{org020bf80}). 

\subsection{ROP-chain embryogenesis}
\label{sec:orgbee2e7b}
\label{org2e86c3c}

Like its predecessor, \gls{roper}, \gls{roper2} maintains distinctions between
genotype and phenotype, on the one hand, and between phenotype and fitness,
on the other. The \emph{phenotype} of an individual, in this context, is its
behaviour during execution. Execution, here, is provided by an emulated
\gls{cpu}, into which an executable binary has already been loaded (the
loading is handled by the \texttt{emu::loader} module, documented in \url{loader.org}).
The individual's \emph{genotype} is serialized into its "natural form" -- a
stack of addresses and machine words, which either dereference to locations
in executable memory (ideally, to \emph{gadgets}) or exist to provide raw numerical
material to be used by the instructions dereferenced. It then loaded into
the CPU's stack memory, and the first address is popped into the \gls{pc}, 
just as it would be in a "stack smashing" attack. The emulator is fired up,
and everything proceeds just as it would in a "wild" \gls{rop} attack.  

\lstset{language=rust,label=org5b68901,caption= ,captionpos=b,numbers=none}
\begin{lstlisting}
<<bring the hatchery's dependencies into scope>>
<<spawn the main hatchery loop>>
<<spawn the subsidiary loops to divy up the workload>>
<<develop phenotypic profiles responding to a range of problem cases>>
  #[inline]
  pub fn hatch(creature: &mut gen::Creature, 
               input: &gen::Input, 
               emu: &mut Engine) -> gen::Pod {
      let mut payload = creature.genome.pack(input);
      let start_addr = creature.genome.entry().unwrap();
      /* A missing entry point should be considered an error,
       * since we try to guard against this in our generation
       * functions.
       */
      let (stack_addr, stack_size) = emu.find_stack();
      payload.truncate(stack_size / 2);
      let _payload_len = payload.len();
      let stack_entry = stack_addr + (stack_size / 2) as u64;
      emu.restore_state();

      /* load payload **/
      emu.mem_write(stack_entry, &payload)
          .expect("mem_write fail in hatch");
      emu.set_sp(stack_entry + *ADDR_WIDTH as u64);

    <<attach hooks for tracking performance>>

      let _res = emu.start(start_addr, 0, 0, 1024);

    <<clean up the hooks>>

      let pod = gen::Pod::new(registers, visited, writelog, retlog);
      pod
  }
\end{lstlisting}


\subsubsection{Hooks for behavioural analysis}
\label{sec:org58481a5}
The phenotype -- or, rather, that aspect of the phenotype that developed
\footnote{In the sense of an embryo, or a photograph.} in this particular execution
-- is then returned in the form of a \texttt{Pod} struct. The fields you see being
passed to the \texttt{Pod} constructor, here, are populated by a series of functions
that have been hooked into the emulator, using \texttt{Unicorn}'s hook API. (The
\texttt{Engine} struct you see at work, here, in \texttt{hatch()}, is defined in the
\texttt{emu::loader} module as well. It is more or less just a convenient 
encapsulation of a \texttt{Unicorn} \gls{cpu} emulator, tailored to ROPER2's needs.)

The hooks are constrained, in type, in two important ways: they must be
\texttt{'static} closures, implementing the \texttt{Fn} trait, and their signature is
fixed in advance. They can mutate data when they are called, so long as
that data has been suitably massaged into reference-counting, internally
mutable cells, but they can't return values. 

The list of hooks used is almost certain to grow, but for the time being
they collect data pertaining to
\begin{itemize}
\item the execution path of the phenotype through memory, in the form of
a vector of addresses visited, along with some useful information
concerning the size of the instructions executed (to facilitate
disassembly, when we come to analyse that path) and the hardware
mode (which, in some \glspl{isa}, such as \gls{arm}, can change at
runtime);
\item the \emph{return} instructions hit. This could be gleaned from the
execution path, but it's much more efficient to track the information
separately. Return-type instructions are of special interest, since
they represent the most basic form by which a \gls{rop} chain can
maintain control over complex execution flows. This is information
that we can profitably put to use in the fitness functions.
\item the valid \emph{writes} performed by the individual, tracked by instruction
address, the destination address of the write, and the size of the
data written.
\end{itemize}

After the execution, we need to clean up the hooks, since they interact
with data structures that will soon be falling out of scope, and we don't
want that data to be unnecessarily held in memory, or to have an accumulating
series of hooks cluttering up and slowing down execution in subsequent runs.

\lstset{language=rust,label=org7617dba,caption= ,captionpos=b,numbers=none}
\begin{lstlisting}
/* Now, clean up the hooks */
match visit_hook {
    Ok(h) => {
        emu.remove_hook(h).unwrap();
    }
    Err(e) => {
        println!("visit_hook didn't take {:?}", e);
    }
}
match mem_write_hook {
    Ok(h) => {
        emu.remove_hook(h).unwrap();
    }
    Err(e) => {
        println!("mem_write_hook didn't take {:?}", e);
    }
}
match ret_hook {
    Ok(h) => {
        emu.remove_hook(h).unwrap();
    }
    Err(e) => {
        println!("ret_hook didn't take: {:?}", e);
    }
}
match indirect_jump_hook {
    Ok(h) => {
        emu.remove_hook(h).unwrap();
    }
    Err(e) => {
        println!("indirect_jmp_hook didn't take: {:?}", e);
    }
}

/* Get the behavioural data from the mutable vectors */
let registers = emu.read_general_registers().unwrap();
let vtmp = visitor.clone();
let visited = vtmp.borrow().to_vec().clone();
let wtmp = writelog.clone();
let writelog = wtmp.borrow().to_vec().clone();
let rtmp = retlog.clone();
let retlog = rtmp.borrow().to_vec().clone();
\end{lstlisting}

\subsubsection{Dealing with multiple problem cases}
\label{sec:orgcac06f7}

Depending on the task at hand, the phenotypic profile that we're
interested in evaluating may need to include the responses of the
individual to a variety of inputs, exemplars, environmental states,
etc. It's simple enough to treat cases where the problem space \emph{isn't}
multiple as a singleton, and so it fits comfortably enough within
this scheme. 

The \texttt{hatch} function is therefore dispatched by another, called
\texttt{hatch\_cases}, which is little more than a \texttt{while} loop, iterating
over the various problem cases associated with the task or environment
of interest.

Since the \texttt{Unicorn} emulator is a foreign struct, implemented in \textbf{C},
there's no easy way to thread this portion of the program. Forcing an
implementation of the \texttt{Send} trait on this struct may expose us to
various race conditions, and other unsafe hazards. 

\subsection{Concurrency plumbing}
\label{sec:orgf244ea0}
\label{org020bf80}

We can nevertheless make great gains in efficiency by spinning
up a set of threads at the beginning of each evaluation phase,
and binding an \texttt{Engine} instance to each thread's scope. The 
main loop of each of those threads is implemented by the function,
\texttt{spawn\_coop}. Rather than collect and return a vector of results
from these evaluations, \texttt{spawn\_coop} maintains a line of communication
back to the caller of the function that called it, in the form of
a \texttt{channel} (specifically, a \texttt{Creature} \texttt{channel}). 

The concurrency paradigm being used here is more or less "the actor
model" of concurrency. There is no shared memory, and when one of our
"actors" (hatcheries or coops) takes possession of a \texttt{Creature}, it
does so uniquely. No mutexes or reference counters are needed to protect
the \texttt{Creature} from race conditions, since it never needs to be in the
hands of two actors at the same time. Instead, we just pass \emph{ownership}
of the \texttt{Creature} from actor to actor -- and thanks to Rust's exquisite
ownership system, this is just a matter of transferring a handful of
machine words. No copying or cloning is needed.\footnote{This is the concurrency model used throughout \gls{roper2}. The only actor
  that takes a clone of a \texttt{Creature}, rather than temporarily seizing ownership
  of it, is the \texttt{logger} actor, which performs statistical analysis on the
  population stream, and logs data to files. This is done so to avoid having
  the logger's relatively expensive operations block the pipeline, and for
  this, skimming off a stream of clones is a small price to pay. The upshot,
  as we'll see, is that the logger needs no return channel. The trip to the 
  agent is one-way, and the clone is dispensed with afterwards.}

The threads are spawned and dispatched by another looping
thread, which is spawned, in turn, by the \texttt{spawn\_hatchery} function.

This function returns almost immediately when called, 
bearing three values to its caller: 

\begin{itemize}
\item \texttt{into\_hatch\_tx}, which is the \texttt{Sender} end of a channel that can be used 
to transmit individuals (of type \texttt{Creature}, which at the time of arrival 
are little more than genomes in hollow shells, whose phenotypes have not 
yet been brought to maturity) to the \emph{genome \(\rightarrow\) phenome map}
\item \texttt{from\_hatch\_rx}, which is the channel on which the caller (or some thread
delegated by the caller) listens for the creatures to return, now developed
into mature phenotypes,
\item \texttt{handle}, the \texttt{JoinHandle} of the thread, which will be used to join the
main hatchery thread.
\end{itemize}


\subsection{Hatchery dependencies}
\label{sec:org3fbb092}
\end{document}